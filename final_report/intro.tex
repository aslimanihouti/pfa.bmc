\subsection*{Contexte}
Braille Music Compiler (BMC) est un outil de conversion du braille
musical principalement destiné à l'usage de personnes aveugles ou
malvoyantes. Il est actuellement à l'état de prototype, il est en
cours de développement par Mario Lang, un autrichien non-voyant
passionné par la musique et l'informatique.

L'idée de développer un tel outil lui est venue lorsqu'il voulu jouer
des partitions braille de musique classique. Souvent, il n'était pas
sûr de bien interpréter certaines notes et s'est vite rendu compte que
cette notation peut parfois porter à confusion. La notation Braille
musicale est, en effet, assez complexe et peut parfois sembler ambiguë
d'où l'utilité d'un programme qui servirait de référence en permettant
d'écouter la note jouée.
  
Seulement la plupart des programmes disponibles sur le marché qui vont
dans ce sens sont assez chers, avec un coût moyen de 600 euros, et ne
sont en général compatibles qu'avec Windows, ce qui limite leur
utilisation.
 
M. Lang commença par travailler sur \textit{FreeDots}, un programme
capable de convertir des partitions \textit{noires} de musique en
partitions braille. Encouragé par le succès que rencontra ce premier
programme, il décida ensuite de se lancer dans la transcription
inverse du braille musical, ce qui donna naissance au Braille Music
Compiler. \\
  
L'amélioration de l'utilisation du BMC notamment par l'implémentation
d'une interface graphique nous a alors été proposée par Samuel
Thibault, chercheur à l'INRIA et proche collaborateur du développeur
du BMC.
  
  
 
\subsection*{Problématique}
Le Braille Music Compiler est encore en cours de développement. En
effet, l'exécution de sa version partielle au début du projet requiert
une certaine connaissance des outils informatiques, chose qui rendait
difficile son utilisation par un usager lambda.

L'intérêt de ce projet est donc d'améliorer cet outil de façon à ce
qu'il devienne accessible à un plus large public. La mise en oeuvre
d'une interface graphique permet en effet de rendre son utilisation
plus intuitive.

Étant principalement dédié à l'usage de personnes non(ou
mal)-voyantes, l'interface graphique implémentée doit donc répondre à
un certain nombre de critères en termes d'accessibilité, elle se veut
simple et fonctionnelle.  De plus, nous avons porté attention à ce que
cette interface comporte aussi des fonctionnalités d'édition et de
modifications de fichiers écrits en braille musical.
   
Le BMC a aussi pour ambition de faciliter la collaboration entre
voyants et non-voyants. En ce sens, l'interface affiche à la fois les
partitions de musique écrites en braille et celles en \textit{noir}
permettant ainsi à chacun des collaborateurs de pouvoir se repérer par
rapport aux partitions de l'autre.

En plus de cette première tache Mario Lang nous a demandé dans un
second temps de nous interesser à la partie backend de son programme
pour enrichir les fonctionnalités de son programme.
