%% Organisation du travail
\subsection{Méthode Scrum}
%% importance des tâches/ périodes de 3 semaines / selection des tâches / encadrant pédagogique
Le responsable pédagogique de ce pfa, monsieur Ta, a préféré au classique cahier des charges une utilisation des méthodes agiles, plus précisément de la méthode \textit{scrum}.  Cette méthode semble être efficace puisque nous avons aujourd'hui atteint en grande partie nos objectifs. En effet, malgré des périodes de bas régime, les dates de rendus périodiques entraînent un certain rythme. Enfin, cette méthode nous prépare à l'entreprise puisque de nos jours, la plupart d'entre elles utilisent cette méthode.



\subsubsection*{Le Product Backlog}
La méthode \textit{scrum} place tous les collaborateurs du projet sur un pied d'égalité. Il faut tout d'abord créer un Product Backlog qui correspond à toutes les tâches à effectuer jusqu'à la finalisation du projet.
Ce document doit être validé avec le client, puis ce dernier attribue des notes aux tâches, aboutissant à un ordre de priorité.

\subsubsection*{Un Sprint Backlog}
Ces dernières sont ensuite réparties en Sprint Backlog. Ce sont des périodes courtes (quelques semaines) durant lesquelles l'équipe travaille sur certaines tâches définies au départ. Nous avons découpé notre Product Backlog en 4 Sprints.
À la fin de chaque Sprint, l'équipe présente sont travail sous forme concrète au client, sous forme d'un exécutable, de maquettes ou toute autre forme.
Ceci est fait pour éviter au projet de partir dans une mauvaise direction et permet au client, dont les envies peuvent varier, de prendre part activement au projet.
La fin d'un Sprint laisse place à une nouvelle réunion où sont définis les objectifs du Sprint suivant.

\subsubsection*{Les moyens de communication}
Afin de pouvoir travailler de façon efficace en groupe, il est nécessaire de disposer d'outils spécialisés dans la gestion de projet.
Nous avons mis en place trois de ces outils.
Pour faciliter la communication au sein du groupe nous avons créé la mail-liste \textit{pfa.bmc@listes.enseirb-matmeca.fr}. Cette liste est utilisable par tout le monde (membres du groupe, client, résponsable pédagogique) et redirige vers les membres du groupe.
Pour gérer la méthode \textit{scrum} nous communiquons via une collection \textit{Google docs} qui se compose essentiellement du Product Backlog, du détail des Sprint Backlog, du compte-rendu des réunions avec l'encadrant et d'un tableau blanc.
Enfin nous avons mis en place un répertoire de travail sur \textit{Github} accessible en écriture aux membres du projet et visible par tous. Celui-ci est accessible à l'adresse \textit{https://github.com/ddallago/pfa.bmc}.

